% Created 2019-12-05 Thu 08:23
% Intended LaTeX compiler: pdflatex
\documentclass[11pt]{article}
\usepackage[utf8]{inputenc}
\usepackage[T1]{fontenc}
\usepackage{graphicx}
\usepackage{grffile}
\usepackage{longtable}
\usepackage{wrapfig}
\usepackage{rotating}
\usepackage[normalem]{ulem}
\usepackage{amsmath}
\usepackage{textcomp}
\usepackage{amssymb}
\usepackage{capt-of}
\usepackage{hyperref}
\usepackage[margin=2cm]{geometry}
\author{Mads Lindskou}
\date{\today}
\title{A Bite of Emacs and ESS}
\hypersetup{
 pdfauthor={Mads Lindskou},
 pdftitle={A Bite of Emacs and ESS},
 pdfkeywords={},
 pdfsubject={},
 pdfcreator={Emacs 26.3 (Org mode 9.1.9)}, 
 pdflang={English}}
\begin{document}

\maketitle
\tableofcontents


\section{\href{img/emacs.png}{Emacs}}
\label{sec:orgc4951c6}

\begin{itemize}
\item Text editor (on steroids)

\item Initial release in 1976 (43 years ago!)

\begin{itemize}
\item Richard Stallman began the GNU Emacs in 1984 (free version)
\end{itemize}

\item Stable release 26.2

\item Cross-platform

\item Active community

\item Extensible and customizable
\end{itemize}

\section{How To Use Emacs}
\label{sec:orgb26cfbd}

\begin{enumerate}
\item Learn the basics and be happy.

\item Learn some advanced commands that suit your needs.

\item Constantly look for new commands to speed up editing.

\item Progress to write your own totally new Emacs commands.
\end{enumerate}

\section{The Basics}
\label{sec:org43efa09}

\subsection{Buffers and Mini buffers}
\label{sec:org1384051}

\begin{itemize}
\item A \textbf{buffer} is an interface between Emacs and a file or process.

\begin{itemize}
\item Equivalent to tabs (and panes are the actual content).)
\end{itemize}

\item The \textbf{minibuffer} is a special buffer.

\begin{itemize}
\item Often a window occupying a single line.
\end{itemize}
\end{itemize}

\subsection{Windows and Frames}
\label{sec:org34044fb}

\begin{itemize}
\item A \textbf{window} is a view onto a buffer.

\begin{itemize}
\item Two windows are what you would split screen in other contexts.
\end{itemize}

\item A \textbf{frame} is the `container`.

\begin{itemize}
\item Called a window in most other contexts.
\end{itemize}
\end{itemize}

\subsection{Keybindings}
\label{sec:org6eb7b6d}

\subsubsection{Nomenclature}
\label{sec:org056559d}

\begin{itemize}
\item \texttt{C-<key>}            : Press control and <key>

\item \texttt{C-<key1> C-<key2>}  : Press control and <key1> release and 
press control and <key2>

\item \texttt{M-x} (M = Meta key) : Alt-x -> type a command; e.g. \texttt{list-packages}
\end{itemize}
\subsection{Some Important Ones}
\label{sec:org9a59151}

\begin{enumerate}
\item \texttt{C-g}        : cancel the current command
\item \texttt{C-x b}      : switch to other buffer
\item \texttt{M-<}        : go to beginning of buffer
\item \texttt{M->}        : go to end of buffer
\item \texttt{C-s}        : incremental search
\item \texttt{C-r}        : search backward
\item \texttt{C-e}        : end of line
\item \texttt{C-a}        : beginning of line
\item \texttt{C-x 1}      : destroy all windows but the this
\item \texttt{C-x 2}     : Split horizontally
\item \texttt{C-x 3}     : split vertically
\item \texttt{C-x o}     : move to next window
\item \texttt{C-x k}     : kill buffer
\item \texttt{C-x f}     : find file
\item \texttt{C-x C-b}   : Ibuffer (enhanced buffer overview)
\item \texttt{C-M-s}     : incremental search using regexp
\item \texttt{M-\%}       : replace string
\item \texttt{C-M-\%}     : replace string regexp
\item \texttt{C-x C-c}   : end emacs (useful on remote servers)
\item \texttt{C-h k}     : describe-keybinding
\item \texttt{C-h f}     : describe function
\item \texttt{C-h m}     : man page of the major mode

\item Some not so important ones
\begin{enumerate}
\item \texttt{M-c} : capitalize word (\texttt{M-capitalize-word})
\item \texttt{M-l} : downcase word   (\texttt{M-downcase-word})
\item \texttt{M-u} : upcase word     (\texttt{M-upcase-word})
\end{enumerate}
\end{enumerate}

\href{https://www.gnu.org/software/emacs/refcards/pdf/refcard.pdf}{Cheatsheet}

\href{https://www.masteringemacs.org/article/mastering-key-bindings-emacs}{Make your own keybindings}

\subsubsection{Killing and Yanking}
\label{sec:org845befd}

\begin{itemize}
\item Also known as \uline{cut} and \uline{paste}

\item But more general because of the \textbf{killring}

\begin{itemize}
\item \texttt{C-w} : Kill   (cut)

\item \texttt{M-w} : Saving (copy) to the killring

\item \texttt{C-y} : Yank   (paste)
\end{itemize}

\item The killring can be seen as a \uline{stack} with the 60 most recent kills

\begin{itemize}
\item \texttt{C-y M-y} : Insert from killring
\end{itemize}
\end{itemize}
\subsubsection{Rectangles}
\label{sec:orgcc37234}

\begin{itemize}
\item \texttt{M-x kill-rectangle} (\texttt{C-x r k})

\item \texttt{M-x yank-rectangle} (\texttt{C-x r y})
\end{itemize}

\textbf{Example - Swap columns:}

A   1

B   2

C   3

\subsection{Modes}
\label{sec:org40c4bcb}

\subsubsection{Major Modes}
\label{sec:org1fc9017}

\begin{itemize}
\item Every buffer possesses a \textbf{major mode}.

\item It determines the editing behavior of Emacs while that buffer is current.

\item It is typically some `language-mode` like

\begin{itemize}
\item \textbf{r-mode}

\item \textbf{c++-mode}

\item \textbf{python-mode}

\item \textbf{makefile-mode}

\item \textbf{text-mode}

\item \textbf{markdown-mode}

\item \textbf{pandoc-mode}

\item \ldots{}
\end{itemize}

\item \texttt{M-x <major-mode>} : change the major mode
\end{itemize}

\subsubsection{Minor Modes}
\label{sec:org451b2e6}

\begin{itemize}
\item A buffer can have several \textbf{minor modes}

\begin{itemize}
\item Auto correction

\item Tab Completion

\item Matching parenthesis

\item Macros

\item \ldots{}
\end{itemize}
\end{itemize}

\section{Multiple Cursors}
\label{sec:orgfdcc94d}

\begin{itemize}
\item \url{https://github.com/magnars/multiple-cursors.el}
\end{itemize}

\textbf{Example}

\begin{verbatim}
X 1.000 0.054 
Y 0.054 1.000 
Z 1.000 0.775 
\end{verbatim}

\section{Dired Mode}
\label{sec:orgba52f91}

Dired is a file browsing system within Emacs

\begin{itemize}
\item \texttt{C-x d}            : Open dired mode

\item \texttt{S-\textasciicircum{}}              : Up-directory

\item \texttt{a}                : Enter directory

\item \texttt{q}                : quit

\item \texttt{m}                : mark file

\item \texttt{u}                : unmark file

\item \texttt{d}                : mark for deletion

\item \texttt{x}                : delete files marked for deletion

\item \texttt{S-!}              : apply a function to file

\item \texttt{C-x C-q}          : enter editing mode

\item \texttt{C-c C-c}          : leave editing mode

\item \texttt{M-S-! nautilus .} : open nautilus here:
\end{itemize}

\section{Bookmarks}
\label{sec:org8f87111}

\begin{itemize}
\item \texttt{C-x r m} : Create new bookmark (can be a file or a folder)

\item \texttt{C-x r b} : Go to bookmark

\item \texttt{C-x r l} : List of all bookmarks
\end{itemize}

\section{The init.el File}
\label{sec:orgde01e1c}

\begin{itemize}
\item When Emacs starts, it initialize your configuration file \textbf{.init}

\begin{itemize}
\item located in the \textbf{.emacs} folder.

\item in a fresh install it contains nothing!
\end{itemize}
\end{itemize}

\subsection{Melpa}
\label{sec:org3f9a840}

\begin{itemize}
\item A package repository for Emacs

\begin{itemize}
\item \url{https://melpa.org/\#/getting-started}
\end{itemize}
\end{itemize}

\begin{verbatim}
(require 'package)
(add-to-list 'package-archives
'("melpa-stable" . "https://stable.melpa.org/packages/") t)
(package-initialize)
\end{verbatim}

\begin{itemize}
\item \texttt{M-x package-install <package>}

\item \texttt{M-x list-packages}

\item Consider the \texttt{paradox} package

\begin{itemize}
\item A lot easier to navigate with

\item \texttt{M-x paradox-list-packages}
\end{itemize}
\end{itemize}

\section{YASsnippets}
\label{sec:orgb129547}

\begin{itemize}
\item A for template system for Emacs

\item \texttt{yasnippet-snippets} : Collection of snippets that supports e.g.
\begin{itemize}
\item \texttt{c-mode}
\item \texttt{c++-mode}
\item \texttt{python-mode}
\item \texttt{lisp-mode}
\item \texttt{latex-mode}
\item \texttt{bibtex-mode}
\item \texttt{org-mode}
\item \texttt{markdown-mode}
\end{itemize}

\item \texttt{yas-describe-tables} : Snippets and keys for current mode

\item No official snippets for \texttt{r-mode}
\begin{itemize}
\item But easy to create
\end{itemize}
\end{itemize}

\begin{verbatim}
# -*- mode: snippet -*-
# name: lapply
# key: lap
# group: *apply family
# --
lapply($1, function(x) $0)
\end{verbatim}

\section{AucTex}
\label{sec:org5cdf965}

\begin{itemize}
\item A major mode for using Latex within Emacs

\item Some useful keybindings

\begin{itemize}
\item \texttt{C-c C-c}    : compile, bibtex, view document
\item \texttt{C-c C-e}    : Start an environment (itemize, equation, tabular, etc.)
\item \texttt{C-c RETURN} : Start macro
\item \texttt{C-c (}      : Insert label
\item \texttt{C-c )}      : \ref
\item \texttt{C-c [}      : \cite, \citep, etc.
\item \texttt{S-=}        : Contents
\end{itemize}

\href{ftp://ftp.gnu.org/gnu/auctex/11.82-extra/tex-ref.pdf}{AucTeX cheatsheet}

\item \textbf{\LaTeX{}-math-mode} - support for math formatting

\begin{itemize}
\item \texttt{C-c \textasciitilde{}}  : Toggle
\item \texttt{`}      : Prefix key
\end{itemize}

\item YASsnippets within AucTeX!

\item Minimal setup in init file:
\end{itemize}

\begin{verbatim}
;; Remove sub and superscript sepcial fonting
(setq font-latex-fontify-script nil)
;; Turn on RefTeX in AUCTeX
(add-hook 'LaTeX-mode-hook 'turn-on-reftex)
;; Activate nice interface between RefTeX and AUCTeX
(setq reftex-plug-into-auctex t)
\end{verbatim}

\section{Emacs Speaks Statistics}
\label{sec:orgb236aca}

\begin{itemize}
\item \href{https://ess.r-project.org/}{ESS}

\item Support for various statistical analysis languages

\begin{itemize}
\item \textbf{R}

\item \textbf{Julia}

\item \textbf{SAS}

\item \textbf{Stata}

\item \textbf{JAGS}
\end{itemize}

\item inferior ESS (iESS) mode is the REPL (interactive shell) we use
\end{itemize}

\subsection{R}
\label{sec:org64c2f1e}

\begin{itemize}
\item In init.el put \texttt{(reqiure 'ess-rutils)}

\item Some useful commands
\begin{itemize}
\item \texttt{C-ENTER}     : send line or region to iESS
\item \texttt{C-c C-UP}    : eval buffer till point
\item \texttt{C-c C-DOWN}  : eval buffer from point
\item \texttt{C-c C-f}     : eval function
\item \texttt{C-c C-s}     : switch process
\item \texttt{C-x C-v}     : documentation for object at point
\item \texttt{C-M-a}       : go to beginning of function
\item \texttt{C-M-e}       : go to end of function
\item \texttt{C-UP}        : previous command in REPL
\item \texttt{C-c C-c}     : break process in REPL
\end{itemize}

\item The family of \texttt{ess-rutils} (\texttt{C-c C. <char>}) 
\begin{itemize}
\item \texttt{o} : rdired
\item \texttt{d} : change the current working directory
\item \texttt{r} : list all available pkgs (and intsall some if you want)
\item \texttt{l} : list all local (installed) pkgs
\end{itemize}

\item ESS drop-down menu
\end{itemize}

\href{http://ess.r-project.org/refcard.pdf}{Cheatsheet}

\subsubsection{Debugging}
\label{sec:org81c9a55}

\begin{itemize}
\item \texttt{C-c C-t <char>} : The family of \texttt{ess-dev-map}

\begin{itemize}
\item \texttt{b} : Set breakpoint

\item \texttt{k} : Kill breakpoint

\item \texttt{n} : Next breakpoint

\item \texttt{p} : Previous breakpoint
\end{itemize}

\item Debugging in \texttt{ess-debug-mode-map}

\begin{itemize}
\item \texttt{M-C} : Continue

\item \texttt{M-N} : Nex line

\item \texttt{M-Q} : Quit
\end{itemize}
\end{itemize}

\subsubsection{Package mode}
\label{sec:org0888adf}

\begin{itemize}
\item Integrates nicely with \texttt{devtools} and \texttt{roxygen2}

\item \texttt{M-x ess-r-devtools-<what>}

\begin{itemize}
\item \texttt{create-package}

\item \texttt{load-package}

\item \texttt{check-package}

\item \texttt{test-package}

\item \texttt{document-package}

\item \texttt{install}
\end{itemize}
\end{itemize}

\subsubsection{Controlling buffer display}
\label{sec:org3fe484d}

\begin{itemize}
\item Default is
\end{itemize}

\begin{verbatim}
----------------------------
|            |             |
|            |             |
|            |             |
|   Source   |   Console   |
|            |             |
|            |             |
|            |             |
----------------------------
\end{verbatim}

\begin{itemize}
\item ESS manual section 3.5 - Controlling buffer display
\end{itemize}

\begin{verbatim}
----------------------------
|            |             |
|   Source   |             |
|            |             |
|------------| Environment |
|            |             |
|   Console  |             |
|            |             |
----------------------------
\end{verbatim}

\begin{itemize}
\item \texttt{ess-r-dired} : Bring up the environment buffer
\end{itemize}

\subsubsection{Tags}
\label{sec:org8a286f0}

\begin{itemize}
\item \texttt{C-c C-e C-t} : create tags for directory (inside an \texttt{.R} file)

\item \texttt{M-.}         : navigate to tag (a function)

\item \texttt{M-,}         : return

\item In package mode

\begin{itemize}
\item \texttt{ess-r-devtools-load-package} : all tags available
\end{itemize}
\end{itemize}

\section{Polymode}
\label{sec:org082716b}

\begin{itemize}
\item Several major modes in one buffer

\item \url{https://polymode.github.io/}

\item \url{https://github.com/polymode/poly-R}
\end{itemize}


\subsection{Rmarkdown}
\label{sec:org5083c68}

\begin{itemize}
\item \texttt{poly-markdown+R-mode}

\item Templates

\begin{itemize}
\item \texttt{M-x \textasciitilde{}poly-r-rmarkdown-create-from-template}

\item Or in \texttt{.Rmd} files

\begin{enumerate}
\item Templates from the RMarkdown dropdown menu

\item \texttt{M-n M-m}
\end{enumerate}
\end{itemize}
\end{itemize}

\section{Magit}
\label{sec:org0eb13b6}

\begin{itemize}
\item A major mode for \textbf{git} version control

\item \href{https://www.masteringemacs.org/article/introduction-magit-emacs-mode-git}{Vanilla tutorial}

\item \href{https://www.youtube.com/watch?v=vQO7F2Q9DwA\&feature=youtu.be}{Video tutorial}

\item \href{https://magit.vc/screenshots/}{Screenshots tutorial}

\item Init setup:
\end{itemize}

\begin{verbatim}
(global-set-key (kbd "C-x g") 'magit-status)
\end{verbatim}
\end{document}